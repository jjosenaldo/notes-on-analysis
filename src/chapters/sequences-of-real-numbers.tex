\setcounter{chapter}{2}
\chapter{Sequences of Real Numbers}

\begin{definition}[Limit of a Sequence; Convergence]
	$a$ is a \emph{limit} of the sequence $\seq {x_n}$ if for every $\eps > 0$ the sequence is eventually $\eps$-closed to $a$.
	We say that $\seq{x_n}$ \emph{converges} to $a$, and that $\seq{x_n}$ is \emph{convergent}.
\end{definition}

\begin{theorem}
	Limits of sequences are unique.
\end{theorem}

\begin{theorem}
	If $\lim x_n = a$ then every subsequence of $\seq{x_n}$ converges to $a$.
\end{theorem}

\begin{theorem}
	Every congervent sequence is limited.
\end{theorem}

\begin{fact}
	If a monotone sequence $\seq{x_n}$ has a limited subsequence, then $\seq{x_n}$ itself is limited.
\end{fact}

\begin{theorem}
	Every limited monotone sequence is convergent.
\end{theorem}

\begin{corollary}[Bolzano--Weierstrass Theorem]
	Every limited sequence has a convergent subsequence.
\end{corollary}

\begin{theorem}
	If $b < \lim x_n$, then eventually $b < \seq{x_n}$. Analogously,
	$b > \lim x_n$ implies that $b > \seq{x_n}$ eventually.
\end{theorem}

\begin{corollary}
	If eventually $\seq{x_n} \le \seq{y_n}$, then $\lim x_n \le \lim y_n$. 
	In particular, if eventually $\seq{x_n} \le c$, then $\lim x_n \le c$.
\end{corollary}

\begin{theorem}[Sandwich's Theorem]
		If $\lim x_n = \lim z_n = l$ and eventually $\seqn x \le \seqn y \le \seqn z$, then $\lim y_n = l$.
\end{theorem}

\begin{theorem}
	If $\lim x_n = 0$ and $\seqn y$ is a limited sequence, then $\lim x_ny_n=0$.
\end{theorem}

\begin{fact}
	$\lim x_n = a \iff \lim (x_n - a) = 0 \iff \lim \modu{x_n - a} = 0$.
\end{fact}

\begin{theorem}
	If $\lim x_n = a$ and $\lim y_n = b$, then
	%
	\begin{enumerate}
		\item $\lim (x_n \pm y_n) = a \pm b$.
		\item $\lim x_n y_n = ab$.
		\item $\lim \frac{x_n}{y_n} = a/b$ given that $b \ne 0$.
	\end{enumerate}
\end{theorem}

\begin{fact}
	If $\seqn x > 0$ and $\lim \frac{x_{n+1}}{x_n}<1$ then $\lim x_n=0$.
\end{fact}

\begin{fact}
	If $a > 1$ and $k \in \nats$ then $$\lim\limits_{n\to\infty} \frac{n^k}{a^n} = 
	\lim\limits_{n\to\infty} \frac{a^n}{n!} = \lim\limits_{n\to\infty} \frac{n!}{n^n}
	=0$$
\end{fact}

\begin{fact}
	If $a>0$ then $\lim a^{1/n}=1$.
\end{fact}

\begin{fact}
	$\lim \paren{\frac{n+1}{n}}^n = \sum_{n=0}^{\infty}\frac1n = e$
\end{fact}

\begin{fact}
	$\lim n^{1/n}=1$
\end{fact}

\begin{fact}
	$\lim\frac{x_n+a/x_n}2=\sqrt a, a>0$
\end{fact}

\begin{definition}[Infinite Limits]
	We say that $\lim x_n = \infty$ when for each $M>0$ eventually $\seqn x > M$. 
	Analogously, $\lim x_n = -\infty$ when for each $N<0$ eventually $\seqn x < N$.
\end{definition}

\begin{fact}
	$\lim x_n = -\infty \iff \lim -x_n = \infty$
\end{fact}

\begin{theorem}
	\begin{enumerate}
		\item[]
		\item If $\lim x_n = \infty$ and $\seqn y$ has a lower bound then $\lim (x_n+y_n) = \infty$.
		\item If $\lim x_n = \infty$ and $\seqn y > c$ for some $c>0$ then $\lim x_n y_n = \infty$.
		\item If $\seqn x > c$ for some $c > 0$ and $\seqn y > 0$ then $\lim \frac{x_n}{y_n} = \infty$.
		\item If $\seqn x$ é limitada e $\lim y_n = \infty$ then $\lim \frac{x_n}{y_n}=0$.
	\end{enumerate}
\end{theorem}

\begin{fact}
	$\lim \frac{\log n }{n} = 0$
\end{fact}