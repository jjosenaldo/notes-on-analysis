\setcounter{section}{4}
\section{Some Topological Notions}

\begin{definition}[Interior Point]
	The point $a$ is \emph{interior} to the set $X \subseteq \reals$ when there is some $\eps > 0$ st $\oo{a - \eps}{a + \eps}\subseteq \reals$. The set of interior points of $X$, called the \emph{interior} of $X$, is represented as $\interior X$.
\end{definition}

\begin{definition}[Neighborhood]
	If $a \in \interior X$, then $X$ is a \emph{neighborhood} of $a$.
\end{definition}

\begin{definition}[Open Set]
	$X\subseteq \reals$ is open iff $\interior X = X$.
\end{definition}

%TODO: define 'eventually contained'

\begin{fact}[Limit of a Sequence in Terms of Open Sets]
	$a = \lim x_n$ iff for every open set $A$ containing $a$, $x_n$ is eventually contained in $A$. 
\end{fact}

\begin{theorem}
	Arbitrary unions of opens are open; finite intersections of opens are open.
\end{theorem}

\begin{definition}[Adherent Point]
	$a$ is \emph{adherent} to the set $X \subseteq \reals$ when $a$ is the limit of a sequence of points of $X$.
\end{definition}

\begin{definition}[Closure]
	The \emph{closure} of $X \subseteq R$ is the set of points that adhere to $X$. It is denoted by $\closure X$.
\end{definition}

\begin{definition}[Closed Set]
	A set $X$ is \emph{closed} when $\closure X= X$.
\end{definition}

\begin{definition}[Dense Set]
	Let $X \subseteq Y$. $X$ is \emph{dense} on $Y$ if $Y \subseteq \closure X$.
\end{definition}

\begin{theorem}
	$a$ adheres to $X$ iff every neighborhood of $a$ intersects $X$.
\end{theorem}

\begin{corollary}
	Closures are closed.
\end{corollary}

\begin{theorem}
	A set is closed iff its complement in $\reals$ is open.
\end{theorem}

\begin{theorem}
	Arbitrary intersections of closeds are closed; finite unions of closeds are closed. 
\end{theorem}

%TODO: review this name
\begin{definition}[Cision]
	The sets $A$ and $B$ are a \emph{cision} of $X = A \cup B$ if
	$A \cap \closure B = \closure A \cap B = \emptyset$.
\end{definition}

%TODO: understand the demonstration of this fact
\begin{theorem}
	Intervals of the rect only admit the trivial cision.
\end{theorem}

\begin{corollary}
	The only clopen subsets of $\reals$ are $\emptyset$ and $\reals$.
\end{corollary}

\begin{definition}[Accumulation Point]
	$a$ is an \emph{accumulation point} of $X$ if every neighborhood of $a$ intersects $X \setminus \set a$. The set of accumulation points of $X$ is denoted as $\accpoints X$.
\end{definition}

\begin{definition}[Isolated Point]
	$a \in X$ is a point \emph{isolated} from $X$ if $a$ is not an accumulation point of $X$.
\end{definition}

\begin{definition}[Discrete Set]
	A set is \emph{discrete} if all of its elements are isolated points.
\end{definition}

\begin{theorem}
	Given $X \subseteq \reals$ and $a \in \reals$, the following are equivalent:
	\begin{enumerate}
		\item $a \in \accpoints X$
		\item $a \in \closure{X \setminus \set a}$
		\item Every interval centered in $a$ has infinite elements of $X$.
	\end{enumerate}
\end{theorem}

\begin{theorem}
	Every bounded infinite set has an accumulation point.
\end{theorem}

\begin{definition}[Compact Set]
	A set $X\subseteq \reals$ is \emph{compact} if it is limited and closed.
\end{definition}

\begin{theorem}
	A set $X \subseteq \reals$ is compact iff every sequence of points of $X$ has a subsequence that converges to a point of $X$.
\end{theorem}

\begin{fact}
	Every compact set has a mininum and a maximum element.
\end{fact}

\begin{theorem}
	Given a sequence $X_1 \supseteq X_2 \supseteq \dots$ of compact nonempty sets, there exists an element belonging to each $X_i$.
\end{theorem}

%TODO: add the definitions of covering, open covering and subcovering

\begin{theorem}[Borel--Lebesgue]
	Every open covering of a compact set has a finite subcovering.
\end{theorem}